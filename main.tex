\documentclass{beamer}
\usepackage[T1]{fontenc}
\usepackage{biblatex}
\usetheme{Madrid}
\usefonttheme[onlymath]{serif}

% \usepackage{pgfpages}
% \pgfpagesuselayout{2 on 1}[a4paper,border shrink=5mm]
\setbeamertemplate{theorems}[ams style]
\setbeamertemplate{theorema}[numbered]

% other packages
\usepackage{latexsym,amsmath,xcolor,multicol,booktabs}
\usepackage{graphicx,pstricks,listings,stackengine}

\addbibresource{references.bib}

\author{Naveen Maurya}
\title[Berger's Holonomy Classification]{Berger's Classification Theorem on Riemannian Holonomies}
\subtitle{MA 333 Final Presentation}
\institute[IISc]{Indian Institute of Science}
\date{2025 Dec 09}

% defs
\def\cmd#1{\texttt{\color{red}\footnotesize $\backslash$#1}}
\def\env#1{\texttt{\color{blue}\footnotesize #1}}
\definecolor{deepblue}{rgb}{0,0,0.5}
\definecolor{deepred}{rgb}{0.6,0,0}
\definecolor{deepgreen}{rgb}{0,0.5,0}
\definecolor{halfgray}{gray}{0.55}

\lstset{
    basicstyle=\ttfamily\small,
    keywordstyle=\bfseries\color{deepblue},
    stringstyle=\color{deepgreen},
    numbers=left,
    numberstyle=\small\color{halfgray},
    rulesepcolor=\color{red!20!green!20!blue!20},
    frame=shadowbox,
}

% Define a softer light blue for sections
\setbeamercolor{section title}{bg=blue!10, fg=black}

\AtBeginSection[]{
  \begin{frame}[plain]
    \vfill
    \centering
    \begin{beamercolorbox}[sep=10pt,center,rounded=true]{section title}
      \usebeamerfont{title}\insertsection\par
    \end{beamercolorbox}
    \vfill
  \end{frame}
}

\setbeamercolor{headerCol}{fg=black,bg=lightgray}
\setbeamercolor{bodyCol}{fg=white,bg=gray}

\newcommand*{\Hol}{\m{Hol}}
\newcommand*{\hol}{\mathfrak{hol}}
\newcommand*{\End}{\m{End}}
\newcommand*{\R}{\mathbb{R}}
\newcommand*{\Sph}{\mathbb{S}}
\newcommand*{\Spin}{\m{Spin}}
\newcommand*{\GL}{\m{GL}}
\newcommand*{\m}{\mathrm}

\begin{document}

\begin{frame}
    \titlepage
\end{frame}

\begin{frame}{Contents}
    \tableofcontents[sectionstyle=show,subsectionstyle=show/shaded/hide,subsubsectionstyle=show/shaded/hide]
\end{frame}

\section{Introduction}

\begin{frame}{Statement of Berger's Classification}
    \pause
    Let $(M, g)$ be a simply-connected, irreducible, nonsymmetric Riemannian Manifold of dim $m$, then the
    holonomy group $\Hol(p)$ of $M$ is exactly one of the following:
    \pause

    \begin{itemize}
        \item $m = n,\;\;\; \Hol(p) = \m{SO}(n)$
        \item $m = 2n, \;\Hol(p) = \m{U}(n)$
        \item $m = 2n, \;\Hol(p) = \m{SU}(n)$
        \item $m = 4n, \;\Hol(p) = \m{Sp}(n)$
        \item $m = 4n, \;\Hol(p) = \m{Sp}(n)\m{Sp}(1)$
        \item $m = 7,\;\;\; \Hol(p) = \m{G}_2$
        \item $m = 8,\;\;\; \Hol(p) = \Spin(7)$
    \end{itemize}
\end{frame}

\section{Preliminary Facts and Nomenclature}

\begin{frame}{Lie Group Representations}
    $G$ be a compact Lie subgroup of $\m{SO}(n)$.
    Then $G$ has a natural representation/action on $\R^n$.
    The stabilizer of $v \in \R^n$ is called the \textit{isotropy subgroup} of
    $v$, $G_v$.\\
    \pause

    Now let $v \in \R^n$, and let $G.v$ be the orbit of $v$ under $G$.
    Then there is an induced representation of $G_v$ 
    on the normal space $\nu_v(G.v)$, called the \textit{slice representation}. \\
    \pause

    If the slice representation is trivial, we say that $v$ is a \textit{principal vector}.
    The set of principal vectors is open and dense in $\R^n$.
    \pause

    If $M$ is an embedded submanifold of $\R^n$, then then there is a derived
    normal holonomy group $\Hol^\perp(\nabla^\perp)$ wrt the normal connection $\nabla^\perp$.
\end{frame}

\section{Reducible Spaces}

\begin{frame}{Reducible Spaces}
    \only<1-3>
    {\pause
    $(M,g)$ is called a \textit{reducible} space if it is isometric to a nontrivial Riemannian product $(M_1, g_1) \times (M_2, g_2)$.\\
    An \textit{irreducible} space is one which cannot be written as a Riemannian product. \\
    \pause
    Note, Holonomy representation of $(M_1,g_1) \times (M_2,g_2)$ is $\Hol(M_1,g_1) \times \Hol(M_2,g_2)$.
    The converse is also true, but locally!\\
    The global converse is achieved if we have completeness.
    }
    \only<4>{
    \begin{theorem}[de Rham Decomposition]
        Let $(M, g)$ be simply-connected and complete. Then there is a unique decomposition
        upto isometry and permutations
        \[
            (M,g) = \prod_{i=1}^{k}(M_i, g_i)
        \]
        where $(M_i, g_i)$ are complete, simply-connected and irreducible.

        Moreover, the holonomy representation of $\Hol_p(M)$ over $T_pM$ is the 
        product of the representations of $\Hol_{p_i}(M_i)$ over $T_{p_i} M_i$.
    \end{theorem}
    Proof is via distributions.
    }
\end{frame}

\section{Symmetric Spaces}

\begin{frame}{Symmetric Spaces}
    \pause
    \begin{definition}
    A Riemannian space $(M,g)$ is said to be \textit{locally symmetric} if for each $x \in M$,
    there exist isometries $s_x$ defined on some neighbourhood of $x$ such that:
        \[
            s_x(x) = x, \text{ and } s_{x*} = -\m{id}
        \]
    \end{definition}
    \pause
    If these $s_x$ are defined on all of $M$, then $M$ is said to be a
    \textit{globally symmetric}. \\
\end{frame}

\begin{frame}
    We have the following result:
    \begin{theorem}
        Let $(M,g)$ be a Riemannian symmetric space. Then
        \begin{enumerate}
            \item [(i)] $\nabla R = 0$
            \item [(ii)] $M$ is complete.
            \item [(iii)] Involutive isometries $s_x$ are transitive on $M$.
        \end{enumerate}
    \end{theorem}
    \pause
    \only<2>{
        Note, (i) follows because
        \begin{align*}
            \nabla R &= s_{x}^* (\nabla R)\\
            (\nabla R)(X, Y, Z, W) &= s_{x*}(\nabla R) (X, Y,Z,W) \\
            &= s_{x*} (\nabla_{s_{x*}X}R) (s_{x*} Y, s_{x*} Z, s_{x*} W) \\
            &= (-1)^5 \nabla R(X,Y,Z,W)
        \end{align*}
    }
    \only<3>{
        Suppose $x, y \in M, X \in T_pM$ st $y = \exp_x(\epsilon X)$
        be joined by a geodesic $\gamma$ on $[0, \epsilon]$.
        Then consider the extension of $\gamma$ to a larger domain,
        \begin{align*}
            \tilde{\gamma}(t) := \begin{cases}
                \exp_x (\epsilon X) & t \in [0, \epsilon] \\
                s_y(\exp_x((t-\epsilon)X)) & t \in [\epsilon, 2\epsilon]
            \end{cases}
        \end{align*}
        This shows $M$ is complete.
    }
    \only<4>
    {
        Now, suppose $\gamma$ is a geodesic from $x$ to $y$ with $\gamma(2T) = y$.
        (which exists since M is complete) Then note $s_{\gamma(T)}$ is an isometry
        taking $x$ to $y$. This show the involutive isometries are transitive on $M$.
    }
\end{frame}

\begin{frame}{s-Representations}
    \pause
    The final result on symmetric spaces implies $M = G/H$, where 
    $G = \m{Isom}^0(M)$, and $H$ be the isotropy subgroup of a 
    point $o \in M$.

    This follows from the Orbit-Stabilizer Theorem.

    \pause
    \begin{definition}
        The isotropy representation of a semisimple simply-connected symmetric space is called an
        \textit{s-Representation}.
    \end{definition}
    A semisimple symmetric space is $M = G/H$ as before such that $G$ is semisimple.
    
\end{frame}

\begin{frame}{Holonomy and Curvature}
    \pause
    \textit{Ambrose-Singer Theorem} connects the holonomy of a connection 
    with its curvature.
    \pause
    \begin{theorem}[Ambrose-Singer]
        Let $\nabla$ be a connection on $(M,g)$. Then
        \[\hol_p(\nabla) = \m{span}_\R\{P^{-1}_\gamma (F_\nabla \cdot v \land w)_x P_\gamma : \gamma \text{ from $p$ to $x$}\}\]
    \end{theorem}
    \pause
    For Levi-Civita connection,
    \[
        \hol(p) = \m{span}_\R\{P^{-1}_\gamma R_x(X,Y) P_\gamma : \gamma \text{ from $p$ to $x$}\}
    \]
    \pause
    And for symmetric spaces,
    \[
        \hol(p) = \m{span}_\R \{R_p(X,Y) : X,Y \in T_p M\}
    \]
    because $\nabla R = 0$.
\end{frame}

\section{Berger's Holonomy Theorem}

\begin{frame}{Some Results}
    \begin{lemma}\label{thm:span}
        Let $G$ be a compact subgroup of $\m{SO}(n)$ acting on $\R^n$, not transitive on the sphere
        and $v \in \R^n$ be a principal vector. Then $\R^n$ is spanned by the family
        of tangent spaces $\nu_\gamma(t)(G.\gamma(t))$ for some $\xi \in \nu_v(G.v)$,
        $\gamma(t) = v + t \xi, t \in \R$.
    \end{lemma}
\end{frame}

\begin{frame}{Some More Results}
    \begin{theorem}\label{thm:s-rep}
        If $G$, a subgroup of $\m{SO}(n)$ acts on $\R^n$ as an \textit{s-representation},
        then the connected component of the normalizer of $G$ in $\m{SO}(n)$, $N_o(G) = G$.
    \end{theorem}
    This is a standard result which 
\end{frame}

\begin{frame}{Normal Holonomy Theorem}
    We have the following result
    \pause
    \begin{theorem}[Normal Holonomy Theorem]
        Let $M$ be a submanifold of $\R^n$ with $p \in M$.
        Then $\Hol_0^\perp(p)$ acts as an \textit{s-representation}
        on $\nu_p M$ (upto the subspace of fixed points), i.e. there exist decompositions
        \begin{align*}
            \nu_p M = V_0 \oplus V_1 \oplus \cdots \oplus V_k, \\
            \Hol_0^\perp(p) = G_1 \times \ldots \times G_k
        \end{align*}
        such that $G_i$ acts on $V_i$ irreducibly as an isotropy representation
        of a simple symmetric space and trivially on all other $V_j$.
    \end{theorem}
\end{frame}

\begin{frame}{Full submanifolds}
    \begin{definition}
        A submanifold of $\R^n$ is called \textit{full} if it is not contained in any proper
        affine subspace of $\R^n$.
    \end{definition}
    Note, an orbit $G.v$ is full iff $G$ acts irreducibly on $\R^n$.
    \begin{theorem}
        Let $G$ be an orthogonal group acting on $\R^n$ such that the orbit $G.v$ is full.
        Then 
    \end{theorem}
\end{frame}

\begin{frame}{Cartan's Theorem on Existence Totally Geodesic Manifolds}\label{thm:cartan}
    Cartan gave a criterion for submanifolds of arbitrary Riemannian manifolds being totally geodesic.
    \begin{theorem}[Existence of Totally Geodesic Manifolds]
        Let $\tilde{M}$ be a Riemannian manifold, $p \in \tilde{M}$, and $V$ a
        linear subspace of $T_p \tilde{M}$.
        There exists a totally geodesic submanifold $M$ of $\tilde{M}$
        with $p \in M$ and $T_p M = V$ if and only if there exists $\epsilon > 0$
        s.t. for every geodesic $\gamma$ in $\tilde{M}$ with $\gamma(0) = p$, $\dot\gamma(0) \in V \cap B_\epsilon(0)$,
        the Riemannian curvature tensor of $\tilde{M}$ preserves the parallel translate of $V$ along $\gamma$ from $p$
        to $\gamma(1)$.
    \end{theorem}
\end{frame}

\begin{frame}{The Glueing Lemma}\label{thm:glue}
    The following lemma will be useful: \\
    First for any $v \in T_pM$, define $\mathcal{F}_v$ to be the family of subspaces 
    of $T_p M$ all containing $v$, which under the exponential map within injectivity radii form totally geodesic, locally symmetric manifolds.
    \begin{lemma}
        Let $M$ be a Riemannian manifold. Assume that for any given
        $v$ in some dense subset of the Eucliean ball of injectivity radius at $p$,
        the family $\mathcal{F}_v$ spans $T_pM$. Then the involution $s_p$ is an isometry of the ball.
    \end{lemma}
    That is, glueing together locally symmetric spaces gives locally symmetric spaces.
    
\end{frame}

\begin{frame}{}
    \begin{theorem}\label{thm:maj}
        Let $M$ be a Riemannian manifold, $p \in M$ and let $\rho$ be the 
        injectivity radius at $p$. Assume $\Phi = \Hol(p)$ acts irreducibly over
        $T_p M$. Denote $N^v$ = $\exp_p(\nu_v(\Phi.v) \cap B_\rho^E(0))$.
        Then for all $v \in T_p M, v \neq 0$,
        \begin{enumerate}
            \item [(i)] $N^v$ is totally geodesic. Further, $N^v$ splits off the geodesic $\gamma_v$.
            And $Hol_p(N^v) \subseteq \Phi^v$
            \item [(ii)] $fd$
            \item [(iii)] $N^v$ is locally symmetric.
        \end{enumerate}
    \end{theorem}
\end{frame}

\begin{frame}{Berger's Theorem (Finally!)}
    \begin{theorem}
        If the holonomy group of an irreducible Riemannian manifold $M$ is not transitive on the sphere.
        Then $M$ is locally symmetric.
    \end{theorem}
    \pause
    Let $p \in M$. The set of principal vectors of $T_p M$ is open and dense.
    And $\Phi = \Hol(p)$ is not transitive on the sphere.\\
    \pause
    So earlier lemma gives $\gamma_\xi(t) = v + t\xi$ in $\nu_v(\Phi.v)$
    with $\nu_{\gamma_\xi(t)}(\Phi.\gamma(t))$ spanning $T_p M$. \\
    \pause
    From Theorem~\ref{thm:maj}, and the Glueing Lemma~\ref{thm:glue}, we conclude
    that $M$ is locally symmetric at $p$.
\end{frame}

\begin{frame}{Berger's List}
    Since transitive actions have been classified completely
    \pause
    \begin{table}
    \centering
    \begin{tabular}{c c}
        \hline
        Group $G$ & Transitive Action on Sphere \\
        \hline
        $\m{SO}(n)$          & $\Sph^{n-1}$ \\
        $\m{SU}(n)$          & $\Sph^{2n-1}$ \\
        $\m{U}(n)$           & $\Sph^{2n-1}$ \\
        $\m{Sp}(n)$          & $\Sph^{4n-1}$ \\
        $\m{Sp}(n)\m{U}(1)$  & $\Sph^{4n-1}$ \\
        $\m{Sp}(n)\m{Sp}(1)$ & $\Sph^{4n-1}$ \\
        $G_{2}$             & $\Sph^{6}$ \\
        $\m{Spin}(7)$        & $\Sph^{7}$ \\
        $\m{Spin}(9)$        & $\Sph^{15}$ \\
        \hline
    \end{tabular}
    \end{table}
    Some of these cases were later shown to not required to be considered.
    \pause
    This gives the final trimmed list.
\end{frame}

\begin{frame}{Berger's List}
    This finally leaves us with the Berger's list, but each of the classes correspond
    to unique geometries.
    \pause

    Holonomy groups of all possible simply-connected irreducible nonsymmetric manifolds are
    classified as follows:
    \pause
    \begin{table}
    \centering
    \begin{tabular}{| c | c | c |}
        \hline
        Holonomy Group & $\m{dim}(M)$ & Corresponding Geometry \\
        \hline
        $\m{SO}(n)$          & $m = n$  & Riemannian manifold \\
        $\m{U}(n)$           & $m = 2n$ & Kähler manifold \\
        $\m{SU}(n)$          & $m = 2n$ & Calabi-Yau manifold\\
        $\m{Sp}(n)$          & $m = 4n$ & Hyperkähler manifold\\
        $\m{Sp}(n)\m{Sp}(1)$ & $m = 4n$ & Quaternion-Kähler manifold\\
        $G_{2}$              & $m = 7$  & $G_{2}$ manifold\\
        $\m{Spin}(7)$        & $m = 8$  & $\m{Spin}(7)$ manifold\\
        \hline
    \end{tabular}\end{table}
\end{frame}

\begin{frame}{References}
    % This command prints the bibliography
    \printbibliography
    
    % Use \nocite{*} if you want to include ALL entries from the .bib file,
    % even those you didn't cite with \cite{} in the main text.
    \nocite{*}
\end{frame}

\begin{frame}
    \begin{center}
        \Huge{Thanks!}
    \end{center}
\end{frame}

\end{document}